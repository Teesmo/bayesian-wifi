% use sig alternate template
%\documentclass[letterpaper,11pt,twoside,twocolumn]
\documentclass{sig-alternate}

% \setlength{\oddsidemargin}{-0.3cm}
% \setlength{\evensidemargin}{-0.3cm}
% \setlength{\textwidth}{16cm}
% \setlength{\topmargin}{-0.8cm}
% \setlength{\textheight}{23.0cm}

%\usepackage[margin=.70in]{geometry}

%\pdfpagewidth=8.5in
%\pdfpageheight=11in
%\setlength\paperheight {11in}
%\setlength\paperwidth {8.5in}
%\setlength{\textwidth}{7in}
%\setlength{\textheight}{9.25in}
%\setlength{\oddsidemargin}{-.25in}
%\setlength{\evensidemargin}{-.25in}


%include our custom files
% for PSTRicks
%\usepackage[pdf]{pstricks}
%\usepackage[off]{auto-pst-pdf}

%graphics included tikz & pfg
\usepackage{graphicx}
\usepackage{epsfig,epic,eepic}
\usepackage{psfrag}
\usepackage[USenglish,german]{babel}
\usepackage{color,pstricks}

% To allow me to use letters with accent
\usepackage[latin1]{inputenc}

%Footnotes
\usepackage{endnotes}

% Fonts
\usepackage[T1]{fontenc}
\usepackage{times}

% Math packages
%\usepackage{amsmath,amssymb,amsfonts,amsthm}
\usepackage{url}
%\usepackage[ruled,noline]{algorithm2e}

% More compact lists
\usepackage{paralist}

% But then adapt spacing
\usepackage{setspace}
\setstretch{1.3}

%for nice tables
\usepackage{multirow}
\usepackage{ctable}

%for nice item lists
\usepackage{paralist}

%usepackage comment for block comments
\usepackage{comment}

%sub-picture numbering
\usepackage{subfigure}
\usepackage[right]{eurosym}

%pseudocode package
\usepackage{pseudocode}

% for PSTRicks
%\usepackage[pdf]{pstricks}
%\usepackage[off]{auto-pst-pdf}

%graphics included tikz & pfg
\usepackage{graphicx}
\usepackage{epsfig,epic,eepic}
\usepackage{psfrag}
\usepackage[USenglish,german]{babel}
\usepackage{color,pstricks}

% To allow me to use letters with accent
\usepackage[utf8]{inputenc}

%Footnotes
\usepackage{endnotes}

% Fonts
\usepackage[T1]{fontenc}
\usepackage{times}

% Math packages
%\usepackage{amsmath,amssymb,amsfonts,amsthm}
\usepackage{url}
%\usepackage[ruled,noline]{algorithm2e}

% More compact lists
\usepackage{paralist}

% But then adapt spacing
\usepackage{setspace}
\setstretch{1.3}

%for nice tables
\usepackage{multirow}
\usepackage{ctable}

%for nice item lists
\usepackage{paralist}

%usepackage comment for block comments
\usepackage{comment}

%sub-picture numbering
\usepackage{subfigure}
\usepackage[right]{eurosym}

%pseudocode package
\usepackage{pseudocode}



% MAIN DOC
\begin{document}

	\title{Thomspon Sampling for rate ressource allocation in WiFi networks}
	\subtitle{}
	\numberofauthors{5}
	\author{
		\alignauthor
		Bjoern Smedman\\
		    \email{bjornsmedman@gmail.com}
		\and
		\alignauthor
		Thomas H{\"u}hn\\
	    		\affaddr{Technical University of Berlin}\\
			\affaddr{Berlin, Germany}\\
			\email{thomas.huehn@inet.tu-berlin.de}
		\and
		\alignauthor
		Amir Sani\\
			\email{reachme@amirsani.com}
		\and
		Toke Høiland-Jørgensen\\
	    		\affaddr{Dept of Mathematics and Computer Science}\\
			\affaddr{Karlstad University, Sweden}\\
			\email{toke.hoiland-jorgensen@kau.se}
		\and
		\alignauthor
		New Contributors\\
			\email{tbi}
	}

	\maketitle
	\sloppy

	\begin{abstract}
	While there has been extensive theoretical work on sophisticated
	resource allocation algorithms for wireless networks, their applicability
	to WiFi (IEEE 802.11) networks is very limited.
	\end{abstract}

	\keywords{rate control, WiFi, ressource allocation, Thompson Samling, explontation, exploration}

	%%%
	\section{Introduction}
	\label{s:intro}


	%%%
	\section{Related Work}
	\label{s:related}


	%%%
	\section{Designing a Multi Arm Bandit WiFi Controller}
	\label{s:design}

	%%
	\subsection{Design Constraints}
	\label{s:constraints}

        The design of our rate control algorithm specifically targets common
        real-world chipset designs, which imposes certain design constraints. In
        this section we summarise these, and outline the assumptions we make
        about the rate control setting.

        \begin{itemize}
        \item \textbf{Transmission length:} The rate control algorithm can
          choose the maximum allowed transmission length for a given
          transmission, as a number of bytes (subject to limits set by the
          802.11 standards). The driver will then try to build an aggregate that
          fills this maximum size, from the packets queued. The actual
          transmission will often be smaller than the maximum size because the
          queue runs empty, or because the queued packet sizes do not line up
          with the maximum size.

        \item \textbf{Number of rates:} Each station has $N$ supported rates
          available for transmission. Each of these is characterised by the
          three-tuple $(b,m,g)$ where $b$ is the bit rate (corresponding to a
          data encoding), $m$ is the MIMO mode (or multi-antenna configuration)
          and $g$ is the chosen guard interval.

          When the transmission length is kept constant, the transmission time
          decreases with $b$, but not necessarily with $m$. $g$ can be chosen as
          `short' or `long', and is simply a fixed time added to the
          transmission of the actual data.

          In general, lower bit rates correspond to more robust data encoding
          schemes, and so will have a higher success probability at the same
          signal-to-noise ratio. However, because other sources of radio
          noise can interfere with the radio transmission, it is not the case
          that a lower bit rate will always succeed when a higher bit rate does:
          As the transmission takes longer, the probability of a random
          interference during its duration increases. This is most important in
          dense deployments, where there are more sources of interference.

        \item \textbf{Retry chain:} The timing constraints prevent the rate
          control algorithm to run in real time when a packet is ready to be
          transmitted. Instead, the driver will queue up aggregates and pass
          them on to the hardware. Each aggregate built up in this way carries
          with it a \emph{retry chain}, which is chosen by the rate selection
          algorithm. The chain has four ordered slots, i.e.
          $c=((r_1,n_1),(r_2,n_2),(r_3,n_3),(r_4,n_4))$.

          Each slot can specify a rate and a maximum number of retries at that
          rate. The hardware will attempt to transmit the aggregate in the order
          specified by the retry chain, or until a transmission succeeds, and
          then return the number of attempts that were needed to transmit the
          aggregate.

        \item \textbf{Objective:} The objective of the rate selection algorithm
          is to transmit as much data as possible in a given time interval (i.e.
          maximise throughput), but also to minimise variation, and minimise the
          latency of each transmission. The latter points are especially
          important when considering the total number of retries to do per
          aggregate.
        \end{itemize}


	%%
	\subsection{Thompson-Sampling}
	\label{s:minstrel-blues}


	%%%
	\section{Simulation Results}
	\label{s:simulations}

	\subsection{Simulation Set-up}

	\subsection{Validation}

	\subsection{Performance Evaluation}


	%%%
	\section{Measurements Results}
	\label{s:measurements}

	\subsection{Measurement Set-up}

	\subsection{Validation}

	\subsection{Performance Evaluation}


	%%%
	\section{Conclusion}
	\label{s:conclusion}


       % Bibliography
	{
	\bibliographystyle{abbrv}
	\small
	\bibliography{our_references}
	}

	%\appendix
	%\include{appendix}

\end{document}
